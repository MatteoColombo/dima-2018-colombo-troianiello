\chapter{Introduction}
\section{Purpose}
The purpose of this document is to describe all the aspects of the design and implementation of NonSoloLibri, a cross platform application whose aim is to help users to manage their book collections.


\section{Scope}
NonSoloLibri is a small social network application designed for book enthusiast that allows them to organize and review their collection.\\
Users will be able to create libraries and to organize books in them.

Beside the storing functionality, NonSoloLibri has a social part that allows them to connect with other people with a friendship system.
Friends will be able to see the each other collections and wish-lists, so that they can surprise each other with presents.

A market place is available, where every verified user will be able to sell or buy books.
\subsection{Goals}

\begin{enumerate}
    \item Allow the users to organize books in libraries.
    \item Allow the users to manage libraries and books.
    \item Allow the users to view their friends wishlist.
    \item Allow the users to sell and buy books in a market place.
\end{enumerate}


\section{Definitions, Acronyms, Abbreviations}

\subsection{Definitions}
\begin{description}
    \item[Cross-platform:] An application that runs on multiple operative systems, for example on both Android and iOS.
    \item[Flutter:] A cross-platform development SDK written in Dart, to develop for iOS, Android and Web.
    \item[Library:] A shelf characterized by a name, a photo and a list of books. 
    \item[Market Place:] A section of the application where users can sell and buy books.
    \item[Verified user:] A user whose identity has been verified by the system and is allowed to sell books in the market place.   
    \item[Widget:] A Flutter component that can represent anything, from a single layout item, to a whole page.
    \item[Wish-list:] A list of books that a user would like to acquire in the future. 
\end{description}


\subsection{Acronyms}
\begin{description}
    \item[API:] Application Programming Interface.
    \item[IEEE:] Institute of Electrical and Electronic Engineers.
    \item[SDK:] Software Development Kit. 
    \item[UML:] Unified Modeling Language. 
\end{description}
\subsection{Abbreviations}
\begin{description}
    \item[G.x:] The design goal number X.
    \item[R.y:] The design requirement number Y.
    \item[D.z:] The domain assumption number Z.  
\end{description}

\section{Reference Documents}
\begin{itemize}
    \item Android Developers website.
    \item Course slides.
    \item Dart Language documentation.
    \item Firebase documentation.
    \item Flutter documentation.
    \item Material Design website. 
\end{itemize}

\section{Document Structure}
This document is divided in multiple chapters:
\begin{description}
    \item[Chapter 1:] 
    \item[Chapter 2:] 
    \item[Chapter 3:] 
    \item[Chapter 4:] 
    \item[Chapter 5:] 
    \item[Chapter 6:] 
\end{description}