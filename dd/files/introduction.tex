\chapter{Introduction}
\section{Purpose}
The purpose of this document is to describe all the aspects of the design and implementation of NonSoloLibri, a cross platform application whose aim is to help users to manage their book collections.


\section{Scope}
NonSoloLibri is a small social network application designed for book enthusiast that allows them to organize and review their collection.\\
Users will be able to create libraries and to organize books in them.

Beside the storing functionality, NonSoloLibri has a social part that allows them to connect with other people with a friendship system.
Friends will be able to see the each other collections and wish-lists, so that they can surprise each other with presents.

A market place is available, where every verified user will be able to sell or buy books.
\subsection{Goals}

\begin{enumerate}
    \item Allow the users to organize books in libraries.
    \item Allow the users to manage libraries and books.
    \item Allow the users to view their friends wishlist.
    \item Allow the users to sell and buy books in a market place.
\end{enumerate}


\section{Definitions, Acronyms, Abbreviations}

\subsection{Definitions}
\begin{description}
    \item[Cross-platform:] An application that runs on multiple operative systems, for example on both Android and iOS.
    \item[Document Oriented:] Non relational database that uses JSON like documents to save data.
    \item[Flutter:] A cross-platform development SDK written in Dart, to develop for iOS, Android and Web.
    \item[Library:] A shelf characterized by a name, a photo and a list of books. 
    \item[Market Place:] A section of the application where users can sell and buy books.
    \item[Mockito:] Testing tool used to mock classes in Flutter. 
    \item[Verified user:] A user whose identity has been verified by the system and is allowed to sell books in the market place.   
    \item[Widget:] A Flutter component that can represent anything, from a single layout item, to a whole page.
    \item[Wish-list:] A list of books that a user would like to acquire in the future. 
\end{description}


\subsection{Acronyms}
\begin{description}
    \item[API:] Application Programming Interface.
    \item[IEEE:] Institute of Electrical and Electronic Engineers.
    \item[ISBN:] International Standard Book Number.   
    \item[JSON:] JavaScript Object Notation. 
    \item[SDK:] Software Development Kit. 
    \item[SQL:] Structured Query Language.
    \item[UML:] Unified Modeling Language. 
\end{description}
\subsection{Abbreviations}
\begin{description}
    \item[G.x:] The design goal number X.
    \item[R.y:] The design requirement number Y.
    \item[D.z:] The domain assumption number Z.  
\end{description}

\section{Reference Documents}
\begin{itemize}
    \item Android Developers website.
    \item Course slides.
    \item Dart Language documentation.
    \item Firebase documentation.
    \item Flutter documentation.
    \item Material Design website. 
\end{itemize}

\section{Document Structure}
This document is divided in multiple chapters:
\begin{description}
    \item[Chapter 1:] In this chapter it is explained the aim of this document and the project in general.
    \item[Chapter 2:] This chapter cointains the description of the application and the list of its requirements, goals and the assumptions that were made during the development.
    \item[Chapter 3:] This chapter contains the in-details description of the application architecture; all the components and their integration is explained.
    \item[Chapter 4:] In this chapter all the interfaces used in the application are explained. This includes those related to the final users, but also those used during the implementation.
    \item[Chapter 5:] In this chapter it is explained how the application was implemented and tested.
\end{description}